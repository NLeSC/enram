\chapter{Organization of the software}
\label{ch:design-philosophy}


Currently, data from the following stations are included:\\
\\
\begin{longtable}[htb]{ll}
\caption{List of radar stations}
\hline
\textbf{station code} & \textbf{full name}\\
\hline
\endhead
CZBRD & Brdy-Praha, Czech Republic\\
CZSKA & Skalky, Czech Republic \\
FIANJ & Anjalankoski, Finland \\
FIIKA & Ikaalinen, Finland \\
FIKOR & Korpo, Finland \\
FIKUO & Kuopio, Finland \\
FILUO & Luosto, Finland \\
FIUTA & Utaj\"{a}rvi, Finland \\
FIVAN & Vantaa, Finland \\
FIVIM & Vimpeli, Finland \\
FRABB & Abbeville, France \\
FRALE & Al\'{e}ria, France \\
FRAVE & LAvesnois, France \\
FRBLA & Blaisy-Haut, France \\
FRBOL & Boll\`{e}ne, France \\
FRBOR & Bordeaux, France \\
FRBOU & Bourges, France \\
FRCAE & Falaise, France \\
FRCHE & Cherves, France \\
FRCOL & Collobri\`{e}res, France \\
FRGRE & Gr\`{e}zes, France \\
FRMCL & Montclar, France \\
FRMOM & Momuy, France \\
FRMTC & Montancy, France \\
FRNAN & Nancy, France \\
FRNIM & N\^{i}mes, France \\
FRTRO & Arcis-sur-Aube, France \\
HRBIL & Bilogora, Croatia \\
IEDUB & Dublin, Ireland  \\
IESHA & Shannon, Ireland  \\
NLDBL & De Bilt, The Netherlands     \\
NLDHL & Den Helder, The Netherlands \\
NOAND & Andoya, Norway \\
NOBML & Boemlo, Norway \\
NOHAS & Hasvik, Norway \\
NOHGB & Haegebostad, Norway \\
NOHUR & Hurum, Norway \\
NORSA & Rissa, Norway \\
NORST & Rost, Norway \\
NOSTA & Stad, Norway \\
PLBRZ & Brzuchania, Poland \\
PLGDA & Gdansk, Poland \\
PLLEG & Legionowo, Poland \\
PLPAS & Pastewnik, Poland \\
PLPOZ & Pozna\'{n}, Poland \\
PLRAM & Ram\.{z}a, Poland \\
PLRZE & Rzesz\'{o}w, Poland \\
PLSWI & \'{S}widwin, Poland \\
SEANG & \"{A}ngelholm, Sweden \\
SEARL & Stockholm-Arlanda, Sweden \\
SEASE & Ase, Sweden \\
SEHUD & Hudiksvall, Sweden \\
SEKIR & Kiruna, Sweden \\
SEKKR & Karlskrona, Sweden \\
SELEK & Leksand, Sweden \\
SELUL & Lulue\r{a}, Sweden \\
SEOSU & \"{O}stersund, Sweden \\
SEOVI & \"{O}rnsk\"{o}ldsvik, Sweden \\
SEVAR & Vara, Sweden \\
SEVIL & Vilebo, Sweden \\
SILIS & Lisca, Slovenia \\
SKKOH & Kojsovska hola, Slovakia \\
SKMAJ & Maly Javornik, Slovakia\\
\end{longtable}

\needspace{10em}
\section{Directory structure}
The following table lists a selection of the most important files and directories as used in the \mbox{`ENRAMVM'} Virtual Machine.

\begin{longtable}[htb]{lp{7cm}}
\hline
\textbf{Location} & \textbf{Description}\\
(relative to \texttt{/home/wbouten/enram/}) & \\
\hline
\endhead
\texttt{data/}  & conbtains the radar data and derived products \\
\texttt{data/harmonized/bird/}  & contains radar data as HDF5 files after harmonization from zip, gz.tar, gif, etc. \\
\texttt{data/harmonized/cluttermaps/}  & contains cluttermaps for a given station, at different threshold levels of clutter \\
\texttt{data/harmonized/odim/}  & radar data for all stations for the period 15-Aug-2011 to 16-Sep-2011 in ODIM-HDF5 format. \\
\texttt{data/harmonized/statistics/}  & contains output from \texttt{src/process/statistics.pro} \\
\texttt{data/raw/}  & radar data for most stations for the period 15-Aug-2011 to 16-Sep-2011 as provided by individual weather services \\
\texttt{data/stations.txt}  & configuration file containing a list of stations that are part of the data set to be analyzed \\
\texttt{doc/}  & some relevant literature\\
\texttt{doc/readme/}  & the latex files needed for compiling this document\\
\texttt{log/stderr.txt}  &  standard error log\\
\texttt{log/stderr.txt}  & standard output log\\
\texttt{other-data/}  & miscellaneous data (currently not used)\\
\texttt{src/}  & source code main directory\\
\texttt{src/io}  & country-specific i/o routines written in IDL\\
\texttt{src/lib}  & programs and functions from external libraries, written in IDL\\
\texttt{src/process/bird\_call.pro} & IDL program that reads the radial velocity, reflectivity, and cluttermap. and then calls various procedures from Adriaan's library, ultimately resulting in profiles of bird density around a given station.  \\
\texttt{src/process/cluttermap.pro}\footref{fnlabel}  & IDL program that calculates cluttermaps at various thresholds of the statistics data.  Output is written to \texttt{data/harmonized/cluttermaps/}.  \\
\texttt{src/process/flysafe2.pro}  & IDL program that loops over a range of dates and various clutter threshold levels, and then calls \texttt{bird\_call.pro}. \\
\texttt{src/process/histogram.pro}\footref{fnlabel}  & IDL program that loops over a range of dates and stations, and then calls \texttt{statistics.pro}. \\
\texttt{src/process/statistics.pro}  & IDL program that calculates a histogram that serves as a base for constructing cluttermaps later. Output is written to \texttt{data/harmonized/statistics/} \\
\multicolumn{2}{l}{\texttt{src/raw-to-odim-conversion-tools}}\\
 & (incomplete) collection of conversion scripts, mostly written in Bash and C, with which the initial conversion was performed to convert the data from \texttt{data/raw} to the ODIM-HDF5 format in \texttt{data/harmonized/odim}.\\
\multicolumn{2}{l}{\texttt{src/visualization/make\_bird\_density\_profiles.pro}\footref{fnlabel}}\\
   & IDL program that creates figures of bird density as a function of time and altitude.\\
\multicolumn{2}{l}{\texttt{src/visualization/make\_colored\_donut\_maps.pro}\footref{fnlabel}} \\
 & IDL program that creates a figure in which the bird density around each radar is plotted as a colored  donut.\\
\multicolumn{2}{l}{\texttt{src/visualization/make\_ppi\_time\_space\_bird\_rain.pro}\footref{fnlabel}}  \\
& IDL program that creates a map of distributed PPI's of rain and of bird density, at given time intervals. \\
\multicolumn{2}{l}{\texttt{src/visualization/multi\_ppi.pro}\footnote{\label{fnlabel}This program is part of the main data processing pipeline, as defined in the Bash script \texttt{start-processing.sh}.}} \\
   & IDL program that visualizes multiple PPI's.\\
\texttt{src/visualization}  & figure-generating programs\\
\texttt{src/vol2bird} & stores Adriaan Dokter's C language program \\
\texttt{testdata/} & subset of \texttt{data/}, used in integration testing\\
\texttt{visualization-output/}  & directory where all visualization programs place their output figure files\\
\end{longtable}
